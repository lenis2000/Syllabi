\documentclass[oneside,11pt]{amsart}
\usepackage[utf8]{inputenc}%
\usepackage[english]{babel}%
\usepackage{amsmath,amssymb,amsthm,amsfonts}%
\usepackage[unicode]{hyperref}%
\usepackage{mathrsfs,bbm}%
\usepackage{paralist}
\usepackage{color}
\usepackage{longtable}
\usepackage{array}
\usepackage{cleveref}
\newcolumntype{L}[1]{>{\small\raggedright\arraybackslash}m{#1}}
\newcolumntype{T}[1]{>{\footnotesize\raggedright\arraybackslash}m{#1}}
\usepackage{stmaryrd}%
\usepackage{graphicx}
\usepackage[DIV17]{typearea}
\usepackage{multicol,tikz}
\usepackage{datetime}

\usepackage{etoolbox}
\patchcmd{\section}{\scshape}{\Large\itshape\bfseries}{}{}

\usepackage{caption}
\captionsetup{labelformat=empty,labelsep=none}

\synctex=1

\begin{document}

\title[MATH 2310: CALCULUS III]{MATH 2310: CALCULUS III}
\author{Leonid Petrov\\Fall 2024\\Sections 100 and 200 (Mondays and Wednesdays)}
\date{\today, \currenttime. An up to date syllabus is always on \href{https://github.com/lenis2000/Syllabi/blob/master/Syllabus_2310_f24.pdf}{\texttt{GitHub}} (for direct PDF download use \href{https://github.com/lenis2000/Syllabi/raw/master/Syllabus_2310_f24.pdf}{\texttt{this link}}).}
\maketitle

\setcounter{tocdepth}{3}

\section{First things first}

\subsection{Overview}

How do you compute the value of $\pi$ with a dart? How do you make the largest possible box out of a piece of paper? How do Google's Maps know all the distances between any two points? How do you find the direction if you are lost in the universe? Most importantly, how come these are all mathematical questions?

Mathematicians have worked for centuries to answer these and many other questions from various aspects of the real world. This course aims to introduce you to one of the most classic and powerful approaches: the Multivariable Calculus. It studies a wide range of topics involving functions in more than a single variable (two or three in most cases). This knowledge is a product of many generations of great mathematicians, and it is at the center of modern sciences and many areas of mathematics. 

By the end of this course, you will be able to:
\begin{itemize}
	\item 
	Describe basic 3-dimensional objects, such as curves, surfaces, solids, and vector fields in various coordinate systems.
	\item 
	Understand further the concept of differentiation, and adapt its algebraic and geometric interpretations to models in physics, chemistry, economics, and other disciplines you learned or are learning in other courses.
	\item 
	Explain the mathematical meaning of infinitesimal and infinity, and implement these to study properties of 2- and 3-dimensional objects such as length, area, volume, etc.
	\item 
	Relate differentiation and integration in various settings and explore how this can give us insights into real-world applications.
	\item 
	Make concise mathematical arguments about the concepts of the course.
\end{itemize}

\subsection{Basic information}{\ }

\vspace{5pt}

\begin{tabular}{r|l|l}
	\hline
	&\textbf{Section 1} & \textbf{Section 2}
	\\\hline
	\textbf{Class times}&MW TBD; TBD & MW TBD; TBD 
	\\\hline
	\textbf{Discussions}&Tu TBD; TBD
	&
	Tu TBD; TBD
	\\\hline
	\textbf{Final exam}&
	TBD&
	TBD
	\\\hline
\end{tabular}

\vspace{10pt}

\parbox{.5\textwidth}{

\textbf{Instructor:} Leonid Petrov

\textbf{Email:} petrov@virginia.edu

\textbf{Office:} 209 Kerchof Hall

\textbf{Office hours:} Mon 11:30AM-12:30PM, Wed 10:30-11:30AM,
or by appointment}\hspace{30pt}
\parbox{.4\textwidth}{

\textbf{Teaching Assistant:} TBD

\textbf{Office:} TBD

\textbf{Office hours:} TBD

\vspace{15pt}

\textbf{Grader:} TBD}

\vspace{5pt}

\subsection{About the instructor}
I am a professor in the Department of Mathematics at UVA, and I've been here since 2014. My research area is probability theory. More precisely, I am using exact formulas to study large random systems. I also like computer simulations of random systems, some examples are \href{http://faculty.virginia.edu/petrov//blog/2015/02/20/Shapes%20with%20holes/}{\texttt{at this link}} or at my office door.

\subsection{Textbook}

\emph{Calculus} or \emph{Multivariable Calculus}, 9th edition, by James Stewart (earlier editions acceptable modulo confirming consistency with assigned material)

\subsection{Discussion sessions} 

In addition to our in-class meetings on Mondays and Wednesdays, there will be 50-minute discussion sessions on Tuesdays led by the Teaching Assistant. These meetings are highly important for the success in the course, and will consist of:
\begin{itemize}
	\item 
	Group work on due homework (all homework will be due on Sundays). By this time you should have thought about most of the homework problems, and should have some questions. The TA will separate you into different study groups, randomly every session. It's a great opportunity to share thoughts and get to know your fellow students. Even if you have figured out all problems, take this chance to practice presenting your thoughts to your peers. In addition, you may get some different perspective on your solutions, which is always helpful. Also read \Cref{academic_integrity} on collaboration.
	\item 
	Discussion of past homework, to clear up things you didn't understand. Solutions to past homeworks will be made available to you.
	\item Occasional unannounced (``pop'') quizzes (2-4 during the semester), to ensure you have the basic understanding of current topics.
	\item The 2 midterm tests will also be given during discussion sessions.
\end{itemize}

\section{Assessment}

Learning mathematics means \emph{doing} mathematics. 

\subsection{In-class work (5\%)} This includes participation in class activities and discussions.

\subsection{WebAssign homework (20\%)} Weekly online homework assignments to practice concepts and techniques.

\subsection{Written homework (20\%)} Weekly written assignments to develop problem-solving skills and mathematical writing.

\subsection{2 midterm tests (15\% each, 30\% total)}

The midterm tests will have similar taste as homeworks, and will test basic knowledge of the material. These 50-minute tests will be given during discussion sessions.

A two-sided letter size formula sheet, hand-written by yourself, will be allowed on each midterm test and the final exam. Preparing this formula sheet will help you review the material, and paint a systematic picture in your head. I encourage you to collaborate on test preparation, but needless to say that during tests and exams each student must work individually.

The use of calculators is not allowed on midterm tests and the final exam.

\subsection{Final exam (25\%)}
The final exam will be cumulative, but will emphasize topics covered after the last midterm.

\section{Communication} \label{comm} 

We will use Canvas for course announcements, assignment submissions, and general communication. Please make sure you have access to our course Canvas page and check it regularly.

\section{Approximate course schedule}

The homeworks are usually due on Sundays at 11:59pm, and most of the time will be assigned at least a week before the due date. The midterm tests are given during discussion sessions, their solutions are discussed and questions on them are answered on Wednesdays the same week.

\vspace{5pt}

\begin{center}
\begin{tabular}{l|l|l}
	\hline
	\textbf{Week}&\textbf{Topics}&\textbf{Sections}
	\\\hline
	1. 8/26 &\parbox{.55\textwidth}{Curves Defined by Parametric Equations, Calculus with Parametric Curves}& 10.1--10.2 \\\hline
	2. 9/2 &\parbox{.55\textwidth}{Polar Coordinates, Three-Dimensional Coordinate Systems, Vectors}& 10.3, 12.1--12.2\\\hline
	3. 9/9 &\parbox{.55\textwidth}{The Dot Product, The Cross Product} & 12.3--12.4\\\hline
	4. 9/16 &\parbox{.55\textwidth}{Equations of Lines and Planes, Cylinders and Quadric Surfaces}& 12.5--12.6\\\hline
	5. 9/23 &\parbox{.55\textwidth}{Vector Functions and Space Curves, Derivatives and Integrals of Vector Functions}&13.1--13.2 \\\hline
	6. 9/30 &\parbox{.55\textwidth}{Arc Length and Curvature, Motion in Space: Velocity and acceleration}& 13.3--13.4 \\\hline
	7. 10/7 &\parbox{.55\textwidth}{Function of Several Variables, Limits and Continuity}& 14.1--14.2\\\hline
	8. 10/14 &\parbox{.55\textwidth}{Partial Derivatives, Tangent Planes and Linear Approximations}& 14.3--14.4\\\hline
	9. 10/21 &\parbox{.55\textwidth}{The Chain Rule, Directional Derivatives and the Gradient Vector}& 14.5--14.6\\\hline
	10. 10/28 &\parbox{.55\textwidth}{Maximum and Minimum Values, Lagrange Multipliers}& 14.7--14.8\\\hline
	11. 11/4 &\parbox{.55\textwidth}{Double Integrals over Rectangles, Double Integrals over General Regions}& 15.1--15.2\\\hline
	12. 11/11 &\parbox{.55\textwidth}{Double Integrals in Polar Coordinates, Applications of Double Integrals}& 15.3--15.4\\\hline
	13. 11/18 &\parbox{.55\textwidth}{Surface Area, Triple Integrals}& 15.5--15.6\\\hline
	14. 11/25 &\parbox{.55\textwidth}{Triple Integrals in Cylindrical and Spherical coordinates}& 15.7--15.8\\\hline
	15. 12/2 &\parbox{.55\textwidth}{Change of Variables in Multiple Integrals, Vector Fields}& 15.9, 16.1\\\hline
	16. 12/9 &\parbox{.55\textwidth}{Review and Final Exam Preparation}& \\\hline
\end{tabular}
\end{center}

\vspace{5pt}

\textbf{Note:} The schedule is approximate, and the time spent on various topics/sections can change during the semester (however, the schedule will then be updated). Midterm dates will be announced at least two weeks in advance.

\section{Important dates}

\begin{itemize}
	\item August 27, 2024: Courses begin
	\item September 2, 2024: Labor Day (classes held)
	\item October 12-15, 2024: Fall Reading Days (no classes)
	\item November 27-December 1, 2024: Thanksgiving recess
	\item December 6, 2024: Courses end
	\item December 9-17, 2024: Final exams
\end{itemize}

\section{How to succeed in the course}

\subsection{Resources}

Here are some things to keep in mind which will help you succeed in the course:
\begin{itemize}
	\item Textbook has a lot of nice examples, and great pictures which will help you visualize the things you learn.
	\item In addition, there are multiple web resources (Wikipedia, Khan Academy, just Google up!) which can complement the textbook's material and style of exposition.
	\item Math tutoring center provides free tutoring for the Calculus III course. Location and schedule will be announced at the beginning of the semester.
	\item 
	Discussion sessions are highly important for the success in the course, with a chance to collaborate on due homeworks, and ask any questions you might have.
	\item Canvas is a great online tool to ask and answer questions about the course and collaborate with your fellow students.
\end{itemize}

\subsection{Collaboration}

Group work on homework problems is allowed and strongly encouraged (unless otherwise stated for a particular assignment). Discussions are in general very helpful and inspiring. Nevertheless, before talking to others, get well started on the problems, and contribute your fair share to the process. 

When completing the written assignments, everyone must write up his or her own solutions in their own words, cite any reference other than the textbook and class notes. Quotations and citations are part of the Honor Code for both UVa and the whole academic community. 

It is very important that you truly understand the homework solutions you hand in, otherwise you may be unpleasantly surprised by your in-class test results.

If you work together on homework, please write the names of your collaborators on the front.

\label{academic_integrity}

\section{Policies}

\subsection{Late/make up work}
Each assignment will have due date and time. Late assignments are not accepted. There will also be no make up midterm tests. However, if you have special needs, emergency, or unavoidable conflicts, please let me know as soon as possible, so we can arrange a workaround.

\subsection{Honor Code}
The University of Virginia Honor Code applies to this class and is taken seriously. Any honor code violations will be referred to the Honor Committee.

\subsection{Special needs}

All students with special needs requiring accommodations should present the appropriate paperwork from the Student Disability Access Center (SDAC). It is the student's responsibility to present this paperwork in a timely fashion and follow up with the instructor about the accommodations being offered. Accommodations for test-taking (e.g., extended time) should be arranged at least 5 business days before an exam.

\end{document}
