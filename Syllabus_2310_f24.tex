\documentclass[oneside,11pt]{amsart}
\usepackage[utf8]{inputenc}%
\usepackage[english]{babel}%
\usepackage{amsmath,amssymb,amsthm,amsfonts}%
\usepackage[unicode]{hyperref}%
\usepackage{mathrsfs,bbm}%
\usepackage{paralist}
\usepackage{color}
\usepackage{longtable}
\usepackage{array}
\usepackage{cleveref}
\newcolumntype{L}[1]{>{\small\raggedright\arraybackslash}m{#1}}
\newcolumntype{T}[1]{>{\footnotesize\raggedright\arraybackslash}m{#1}}
\usepackage{stmaryrd}%
\usepackage{graphicx}
\usepackage[DIV17]{typearea}
\usepackage{multicol,tikz}
\usepackage{datetime}

\usepackage{etoolbox}
\patchcmd{\section}{\scshape}{\Large\itshape\bfseries}{}{}

\usepackage{caption}
\captionsetup{labelformat=empty,labelsep=none}

\synctex=1

\begin{document}

\title[MATH 2310: CALCULUS III]{MATH 2310: CALCULUS III}
\author{Leonid Petrov\\Fall 2024\\Sections 100 and 200 (Mondays and Wednesdays)}
\date{\today, \currenttime. An up to date syllabus is always on \href{https://github.com/lenis2000/Syllabi/blob/master/Syllabus_2310_f24.pdf}{\textcolor{green!70!black}{\texttt{GitHub}}} (for direct PDF download use \href{https://github.com/lenis2000/Syllabi/raw/master/Syllabus_2310_f24.pdf}{\textcolor{green!70!black}{\texttt{this link}}}). Note: this PDF has green clickable links, like in the previous sentence. This feature only works if you download the PDF first — it won’t work in browser on GitHub.}
\maketitle

\setcounter{tocdepth}{3}

\section{First things first}

\subsection{Overview}

How do you compute the value of $\pi$ with a dart? How do you make the largest possible box out of a piece of paper? How do Google's Maps know all the distances between any two points? How do you find the direction if you are lost in the universe? Most importantly, how come these are all mathematical questions?

Mathematicians have worked for centuries to answer these and many other questions from various aspects of the real world. This course aims to introduce you to one of the most classic and powerful approaches: the Multivariable Calculus. It studies a wide range of topics involving functions in more than a single variable (two or three in most cases). This knowledge is a product of many generations of great mathematicians, and it is at the center of modern sciences and many areas of mathematics. 

By the end of this course, you will be able to:
\begin{itemize}
	\item 
	Describe basic 3-dimensional objects, such as curves, surfaces, solids, and vector fields in various coordinate systems.
	\item 
	Understand further the concept of differentiation, and adapt its algebraic and geometric interpretations to models in physics, chemistry, economics, and other disciplines you learned or are learning in other courses.
	\item 
	Explain the mathematical meaning of infinitesimal and infinity, and implement these to study properties of 2- and 3-dimensional objects such as length, area, volume, etc.
	\item 
	Relate differentiation and integration in various settings and explore how this can give us insights into real-world applications.
	\item 
	Make concise mathematical arguments about the concepts of the course.
\end{itemize}

\subsection{Basic information}{\ }

\vspace{5pt}

\noindent
\hspace*{-0.7cm}\begin{tabular}{r|l|l}
	\hline
	&\textbf{Section 100} & \textbf{Section 200}
	\\\hline
	\textbf{Class times}&MoWe 2:00PM - 3:15PM; Monroe Hall 122 & MoWe 3:30PM - 4:45PM; Monroe Hall 122 
	\\\hline
	\textbf{Discussions}&Tu 6:00PM - 6:50PM; Monroe Hall 122
	&
	Tu 5:00PM - 5:50PM; New Cabell Hall 323
	\\\hline
	\textbf{Midterm 1}&
		Sept 25, 2024, 2:00PM - 2:50PM (Monroe 122)
		&
		Sept 25, 2024, 3:30PM - 4:20PM (Monroe 122)
	\\\hline
	\textbf{Midterm 2}&
	Oct 29, 2024, 6:00 - 6:50PM (Monroe 122)&
	Oct 29, 2024, 5:00 - 5:50PM (New Cabell 323)
	\\\hline
	\textbf{Final exam}&
	Dec 10, 2024, 2:00PM - 5:00PM (MON122)
	&
	Dec 13, 2024, 2:00PM - 5:00PM (MON122)
	\\\hline
\end{tabular}

\vspace{10pt}

\parbox{.5\textwidth}{

\textbf{Instructor:} Leonid Petrov

\textbf{Email:} petrov@virginia.edu

\textbf{Office:} 209 Kerchof Hall

\textbf{Office hours:} Mon 11:30AM-12:30PM, Wed 10:30-11:30AM,
or by appointment (by email; you can make as many appointments as you need). No OH on Sep 23, 25}\hspace{30pt}
\parbox{.45\textwidth}{

\textbf{TA (100):} Ziming Chen

\textbf{Office hours:} Thu 3:30-4:30PM, Ker 123

\textbf{TA (200):} Aoran Wu

\textbf{Office hours:} TBD


\vspace{15pt}

\textbf{Grader:} Suren Kyurumyan}

\vspace{5pt}

\subsection{About the instructor}
I am a professor in the Department of Mathematics at UVA, and I've been here since 2014. My research area is probability theory. More precisely, I am using exact formulas to study large random systems. I also like computer simulations of random systems (\href{https://lpetrov.cc/simulations/2019-04-30-qvol/}{\textcolor{green!70!black}{\texttt{example}}}).

\subsection{Textbook}

\emph{Calculus} or \emph{Multivariable Calculus}, 9th edition, by Stewart et al (ISBN for Webassign: 9780357128930). Earlier editions acceptable modulo confirming consistency with assigned material.

A digital version of the textbook is provided through the WebAssign online homework system, to which you must have access. A physical copy of the text is optional. 

\subsection{Discussion sessions} 

In addition to our in-class meetings on Mondays and Wednesdays, there will be 50-minute discussion sessions on Tuesdays led by the Graduate Teaching Assistant. These meetings are highly important for the success in the course, and will consist of:
\begin{itemize}
	\item 
		Group work on due homework (homework is usually due on Sundays). By Tuesday, you should have thought about most of the homework problems, and should have some questions. The TA will separate you into different study groups, \emph{randomly every session}. It's a great opportunity to share thoughts and get to know your fellow students. Even if you have figured out all problems, take this chance to practice presenting your thoughts to your peers. In addition, you may get some different perspective on your solutions, which is always helpful. Also read \Cref{academic_integrity} on collaboration.
	\item 
	Discussion of past homework, to clear up things you didn't understand. Solutions to past homeworks will be made available to you.
	\item Occasional unannounced (``pop'') quizzes (3-5 during the discussion sessions, and 3-5 during lectures), to ensure you have the basic understanding of current topics.
	\item Midterm 2 will be given during discussion sessions.
\end{itemize}

\section{Assessment}

Learning mathematics means \emph{doing} mathematics. 

\subsection{WebAssign homework (12.5\%)} Weekly online homework assignments to practice concepts and techniques.
\href{https://www.webassign.net/}{\textcolor{green!70!black}{\texttt{WebAssign}}} is an electronic homework delivery system. You will have homework due there almost every week. Note that purchasing WebAssign also gets you an electronic copy of the textbook. Purchasing a hard copy of the textbook is completely optional.

Any student who purchased WebAssign for a previous calculus course at UVA may already have WebAssign access for this course via the same code used previously. Try your code!

If you must purchase WebAssign for Math 2310, you will need to purchase single-term access online through the WebAssign website. If you want a hard copy of the text (not required but could be useful for those of you like me who don't like reading books online) you can buy a used copy from Amazon or someplace similar.

You have free WebAssign access to the text and course homework sets for the first two weeks of class. To access WebAssign:

\begin{enumerate}
    \item Go to \url{https://www.webassign.net/}
    \item Click the gray button on the upper right
		\item Enter our class key (to be posted to Canvas)
\end{enumerate}

\begin{remark}
	The homework on WebAssign is sometimes called ``quizzes'', but everywhere throughout this syllabus the term ``quiz'' refers to in-class quizzes.
\end{remark}


\subsection{Written homework (12.5\%)} Weekly written assignments to develop problem-solving skills and mathematical writing. They are due on Sundays at 11:59PM, on Gradescope (linked from the course Canvas page).

\begin{enumerate}[$\bullet$]
	\item Observe rules of academic integrity.
		Handing in plagiarized work, whether copied from a fellow student or off the web, is not acceptable (\textbf{Exception/TBD}: usage policy of AI tools like chatGPT and computer algebra systems like Wolfram Alpha / Mathematica will be determined after the first lecture; see \Cref{AI_tools}).
    
	\item Homework must be submitted on time. If you need an extension, you have to ask about it before the due date. Late homework will not be accepted.
    
    \item Working in groups on homework assignments is strongly encouraged; however, every student must write their own assignments. See \Cref{academic_integrity} for more details on collaboration.
    
    \item Organize your work neatly. 
There should not be text crossed out.
Do not hand in your rough draft or first attempt.
Papers that are messy, disorganized or unreadable cannot be graded. Use proper English. Write in complete English or mathematical sentences.
    \item Answers should be simplified as much as possible. If the answer is a simple fraction or expression, a decimal answer from a calculator is not necessary. For some exercises you will need a calculator to get the final answer.
    
    \item Answers to some exercises are in the back of the book, so answers alone carry no credit. It's all in the reasoning you write down.
    
    \item Put problems in the correct order.
\end{enumerate}

\subsection{Quizzes (15\% total)}
 There will be unannounced (``pop'') quizzes in discussion sessions and in class. There will be about 6-10 quizzes total. Lowest 2 or 3 quizzes will be dropped (depending on the total number).


\subsection{Two midterm tests (15\% each, 30\% total)}

The midterm tests will have similar taste as homeworks and quizzes, and will test basic knowledge of the material. The first 50-minute test will be in class; the second one is during the discussion session (see the schedule in the beginning of the syllabus).

\subsection{Final exam (30\%)}
The final exam will be cumulative, but will emphasize topics covered after the last midterm.

\subsection*{Allowed and forbidden resources for quizzes, midterms, and the final exam}

Quizzes are open notes, midterms and final are not. At in-class quizzes, you can use your own notes, but not the textbook.

A two-sided letter size formula sheet, hand-written by yourself, will be allowed on each midterm test and the final exam. Preparing this formula sheet will help you review the material, and paint a systematic picture in your head. I encourage you to collaborate on test preparation, but needless to say that during tests and exams each student must work individually.

You can also use a calculator on midterms and the final exam (but \emph{not the quizzes}). You are not allowed to use a cell phone or any other electronic device including laptops during quizzes, midterms, and the final exam. Calculator on the phone is not allowed.
	
\section{Communication} \label{comm} 

We will use Canvas for course announcements, assignment submissions, and general communication. Please make sure you have access to our course Canvas page and check it regularly.


On Canvas, I have enabled the Piazza tool. Please use it for all questions related to the course material, homework, quizzes, and exams. This way, everyone can benefit from the answers. If there is enough activity on Piazza, I will give extra credit to active participants.\footnote{Extra credit points are added to the numerator and denominator of your exam score. Example: If your raw score is 75/100, but you have 8 points, then your final score would be $83/108 \approx 76.9$.}

You are also welcome to ask questions by email (\texttt{petrov@virginia.edu} or
\texttt{lenia.petrov@gmail.com}; both work equally well for me). Response time by email is usually within 24 hours, except on weekends.

\section{Approximate course schedule}

The homeworks are usually due on Sundays at 11:59PM, and most of the time will be assigned at least a week before the due date. Quiz and midterm solutions will be posted soon after, and can be discussed in the discussion sessions.


\vspace{5pt}


\begin{center}
\begin{longtable}{|l|p{0.5\textwidth}|l|}
    \hline
    \textbf{Week} & \textbf{Topics} & \textbf{Sections} \\
    \hline
    1. 8/26 & Curves Defined by Parametric Equations, Calculus with Parametric Curves & 10.1--10.2 \\
    \hline
    2. 9/2 & Polar Coordinates, Three-Dimensional Coordinate Systems, Vectors, The Dot Product* & 10.3, 12.1--12.3 \\
    \hline
    3. 9/9 & The Cross Product, Equations of Lines and Planes, Cylinders and Quadric Surfaces & 12.4--12.6 \\
    \hline
		4. 9/16 & Vector Functions and Space Curves, Derivatives and Integrals of Vector Functions, Arc Length and Curvature,
		 Motion in Space: Velocity and Acceleration& 13.1--13.4 \\
    \hline
    5. 9/23 & Function of Several Variables, Limits and Continuity & 14.1--14.2 \\
    \hline
    6. 9/30 & Partial Derivatives*, Tangent Planes and Linear Approximations, The Chain Rule & 14.3--14.5 \\
    \hline
    7. 10/7 & Directional Derivatives and the Gradient Vector, Maximum and Minimum Values & 14.6--14.7 \\
    \hline
		8. 10/14 (break) & Lagrange Multipliers, Double Integrals over Rectangles & 14.8, 15.1 \\
    \hline
    9. 10/21 & Double Integrals over General Regions, Double Integrals in Polar Coordinates & 15.2--15.3 \\
    \hline
    10. 10/28 & Applications of Double Integrals, Surface Area & 15.4--15.5 \\
    \hline
    11. 11/4 & Triple Integrals, Triple Integrals in Cylindrical Coordinates & 15.6--15.7 \\
    \hline
    12. 11/11 & Triple Integrals in Spherical Coordinates, Change of Variables in Multiple Integrals & 15.8--15.9 \\
    \hline
    13. 11/18 & Vector Fields, Line Integrals, The Fundamental Theorem for Line Integrals & 16.1--16.3 \\
    \hline
    14. 11/25
		(break) & Green's Theorem, Curl and Divergence & 16.4--16.5 \\
    \hline
		15. 12/6 & 
		Parametric Surfaces and Their Areas, Surface Integrals,
		Stokes' Theorem, The Divergence Theorem
		& 16.6--16.9 \\
		\hline
    \multicolumn{3}{l}{* Skip section on Direction Angles and Direction Cosines in 12.3}\\
	\multicolumn{3}{l}{* Skip partial differential equations in 14.3} \\
    % \multicolumn{3}{l}{** Covered in discussion} \\
\end{longtable}
\end{center}

\vspace{5pt}

\textbf{Note:} The schedule is approximate, and the time spent on various topics/sections can change during the semester (however, the schedule will then be updated).

\section{How to succeed in the course}

\subsection{Resources}

Here are some things to keep in mind which will help you succeed in the course:
\begin{itemize}
	\item Textbook has a lot of nice examples, and great pictures which will help you visualize the things you learn.
	\item In addition, there are multiple web resources (Wikipedia, Khan Academy, just Google up!) which can complement the textbook's material and style of exposition.
	\item Use AI tools (chatGPT, Claude, UVA \href{https://virginia.service-now.com/its?id=itsweb_kb_article&sys_id=dbe41947dbe3f91066d98f38139619db}{\textcolor{green!70!black}{\texttt{free microsoft copilot}}}, etc.) to help you understand the material better, or to fix your grammar / English, if needed. Other allowed uses of AI tools will be determined after the first lecture (see \Cref{AI_tools}).
	\item Free math collaborative learning center at UVA provides help for the Calculus III course. See \href{https://math.virginia.edu/undergraduate/MCLC/}{\textcolor{green!70!black}{\texttt{here}}}.
	\item 
	Discussion sessions are highly important for the success in the course, with a chance to collaborate on due homeworks, and ask any questions you might have.
	\item Canvas/Piazza is a great online tool to ask and answer questions about the course and collaborate with your fellow students.
\end{itemize}

\subsection{Collaboration}

Group work on homework problems is allowed and strongly encouraged (unless otherwise stated for a particular assignment). Discussions are in general very helpful and inspiring. Nevertheless, before talking to others, get well started on the problems, and contribute your fair share to the process. 

When completing the written assignments, everyone must write up his or her own solutions in their own words, cite any reference other than the textbook and class notes. Quotations and citations are part of the Honor Code for both UVa and the whole academic community. 

It is very important that you truly understand the homework solutions you hand in, otherwise you may be unpleasantly surprised by your in-class test results.
If you work together on homework, please write the names of your collaborators on the front.
\label{academic_integrity}

\subsection{Use of AI tools and computer algebra systems like Wolfram Alpha / Mathematica} 
\label{AI_tools}

Policy will be determined after the first lecture.


\section{Policies}

\subsection{Late/make up work}
Each assignment will have due date and time. Late assignments are not accepted. If you have special needs, emergency, or unavoidable conflicts, please let me know as soon as possible, so we can arrange a workaround --- especially for midterms and the final exam. There are no make up quizzes.

\subsection{Honor Code}
The University of Virginia Honor Code applies to this class and is taken seriously. Any honor code violations will be referred to the Honor Committee. Refer to \Cref{academic_integrity,AI_tools} on what constitutes an honor code violation in this course regarding collaboration and the use of AI tools.

\subsection{Special needs}

All students with special needs requiring accommodations should present the appropriate paperwork from the Student Disability Access Center (SDAC). It is the student's responsibility to present this paperwork in a timely fashion and follow up with the instructor about the accommodations being offered. Accommodations for test-taking (e.g., extended time) should be arranged at least 5 business days before an exam. 
If you need accommodation for midterms and the final exam, sign up with SDAC well in advance --- these rooms fill up quickly.

\subsection{Recording for a personal study}

Class sessions for this course may be audio recorded as a
reasonable accommodation for a disability for the student’s
own personal study and review. These audio recordings will
be deleted at the end of the semester. Recordings will not
be reproduced, shared with those not enrolled in the class,
nor uploaded to other online environments.


\end{document}
