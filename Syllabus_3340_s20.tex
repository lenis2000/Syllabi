\documentclass[oneside,11pt]{amsart}
\usepackage[utf8]{inputenc}%
\usepackage[english]{babel}%
\usepackage{amsmath,amssymb,amsthm,amsfonts}%
\usepackage[unicode]{hyperref}%
\usepackage{mathrsfs,bbm}%
\usepackage{paralist}
\usepackage{color}
\usepackage{longtable}
\usepackage{array}
\newcolumntype{L}[1]{>{\small\raggedright\arraybackslash}m{#1}}
\newcolumntype{T}[1]{>{\footnotesize\raggedright\arraybackslash}m{#1}}
\usepackage{stmaryrd}%
%\usepackage{refcheck}
\usepackage{graphicx}
\usepackage[DIV14]{typearea}
\usepackage{multicol,tikz}
\usepackage{datetime}
\usepackage{cleveref}

\usepackage[shadow]{todonotes}

\usepackage{etoolbox}
\patchcmd{\section}{\scshape}{\Large\itshape\bfseries}{}{}

\usepackage{caption}
\captionsetup{labelformat=empty,labelsep=none}

\hypersetup{
  colorlinks=true,
  linkcolor=blue!50!red,
  urlcolor=green!60!black
}

%%%%%%%%%%%%%%%%%%%%%%%%%%%%%%%%%%%%%%%%%%%%%%%%%%%%%%%%%%%%%%%%%%%%%%%%%%%%%%%%%%%%%%%%
\synctex=1
%%%%%%%%%%%%%%%%%%%%%%%%%%%%%%%%%%%%%%%%%%%%%%%%%%%%%%%%%%%%%%%%%%%%%%%%%%%%%%%%%%%%%%%%
%%%%%%%%%%%%%%%%%%%%%%%%%%%%%%%%%%%%%%%%%%%%%%%%%%%%%%%%%%%%%%%%%%%%%%%%%%%%%%%%%%%%%%%%
\newcommand{\score}[1]{\textit{#1}\addtocounter{totalscore}{#1}}
\newcommand{\razdel}[1]{\smallskip\underline{\textbf{#1:}}\smallskip}

\newcommand{\note}[1]{{\sf{}\color{blue}(#1)}}

\begin{document}

\title[MATH 3340: COMPLEX VARIABLES WITH APPLICATIONS]{MATH 3340: COMPLEX VARIABLES WITH APPLICATIONS}
\author{Leonid Petrov\\Spring 2020}
\date{Compiled on \today, \currenttime{} (in whatever timezone I was at that time).\\An up to date syllabus is always on \texttt{GitHub} at \url{https://github.com/lenis2000/Syllabi/blob/master/Syllabus_3340_s20.pdf}. For direct PDF download use \href{https://github.com/lenis2000/Syllabi/raw/master/Syllabus_3340_s20.pdf}{\texttt{this link}}.
	\LaTeX{} source with \textit{changes} to the syllabus is \href{https://github.com/lenis2000/Syllabi/blob/master/Syllabus_3340_s20.tex}{\texttt{here}}
(click ``History'').
\\Note that this PDF has green clickable links.}
\maketitle

\section{Complex variables}

Complex analysis is a central part of 
Mathematics. Many concepts work easier and much more natural
in the complex setup. For example, if a function $f(z)$ 
of the complex variable $z$
has one 
derivative at a point $z_0$, then it has infinitely many derivatives,
and possesses a power series (Taylor) expansion at $z_0$, which converges to our function. 
Compare this with the “bad” behavior of the function 
$f(x)=e^{-1/x}$ for $x>0$ (and $f(x)=0$ for $x\le 0$) of the real variable $x$,
which has infinitely many derivatives, but whose Taylor series at $0$ is 
identically zero.
The course is centered around the basics of the theory of functions of a
single complex variable.

\medskip

After taking this course, you 
will be able to solve problems and understand the 
basics of
complex numbers, analytic functions, 
complex integration, Cauchy formulas, power series, 
residues, and conformal mappings.
Moreover, you will learn how to apply these tools to 
other parts of Mathematics, and to some physical models.

\begin{figure}[h]
	\includegraphics[height=.45\textwidth]{img/complex_f.jpg}
	\caption{Real part of a particularly complex function}
\end{figure}

\subsection*{Prerequisites}

Good command of single and multivariable calculus at the level of MATH 1310, 1320, and 2310.

\section{Necessary information}
\bigskip

\textbf{Class times:}   TuTh 12:30PM - 1:45PM in
\emph{New Cabell 309}

\medskip


\textbf{Exams:} Please do not make travel plans which conflict
with the midterms or the final exam.
\begin{itemize}
	\item \textbf{Midterm 1:} In-class on Thursday, February 6 (class time, New Cabell 309).
	\item \textbf{Midterm 2:} In-class on Tuesday, April 7 (class time, New Cabell 309).
	\item \textbf{Final exam:} Tuesday, May 5, 2-5 (New Cabell 309).
\end{itemize}

\medskip

\textbf{Instructor:} Leonid Petrov
\medskip

\textbf{Email:} \email{petrov@virginia.edu} or \email{lenia.petrov@gmail.com}
\medskip

\textbf{Office:} 209 Kerchof Hall
\medskip

\textbf{Teaching Assistant:} \colorbox{yellow}{\parbox{.7\textwidth}{TBA}}
\medskip

\textbf{Office hours:}
The default times I am in office are Tuesdays and Thursdays, 9:30-10:30,
except the weeks when I'm \href{https://lpetrov.cc/2019/05/travel-2020/}{\texttt{traveling}}.

You are welcome to make an appointment and meet outside the usual office hours. 
For this, please use the online tool located at
\url{https://lpetrov.cc/teaching/}. (I am automatically available during office hours --- 
and you cannot schedule appointments online for those times.)
You can make as 
many appointments as you want.

\medskip

\textbf{Course webpage:}
I will set up a collab page for homework submissions and course materials.

\section{Course materials}

The textbook is “\emph{Fundamentals of Complex Analysis}” (3rd edition)
by Saff and Snider, Pearson, ISBN-10: 0139078746.
We will discuss material from Chapters 1--6, and selected topics from Chapters 7--8.

\section{Assessing your learning}

Learning mathematics means \emph{doing} mathematics: during class meetings, on your own, and in groups. 
In this course, doing mathematics mainly amounts to solving problems. 
Below are the concrete aspects which are assessed in this course:

\subsection{Homework}

\colorbox{red!60!white}{\parbox{.7\textwidth}{this section}}

Weekly homework will consist of
problems aligned with lectures
and of other exploratory theoretical topics,
to help you practice and enrich the material presented in class.
Putting an adequate effort into solving the homework
problems and
communicating your solutions clearly is
of paramount importance for your learning.
Level of homework problems ranges from easy to very difficult;
hints will be given for the most challenging problems.
The homeworks are usually due on
Thursdays, and will be assigned at least a week before the due
date.

Homework solutions are posted soon after the
homework deadline, so late work cannot be accepted.
The lowest homework grade will be dropped.

\subsection*{Homework submission guidelines --- strictly enforced}
The homework \textbf{must be submitted only on Collab} (i.e., hard copies are not accepted).
Take pictures or scan your work,
make sure it's readable,
put it into a \emph{single PDF file with correct orientation},
and upload it before the deadline.
Please also \textbf{put your problems in order}, indicating clearly which problems you're skipping --- this will greatly help with the grading.

Submitting work like this has many benefits:
(1) you retain a paper copy to
prepare for tests;
(2) your submitted work is never misplaced or lost, and there is a digital trail;
(3) the grading will be much faster and will allow me to immediately
incorporate my impressions of homework solutions into in-class
discussions.

If you have any trouble submitting homework online, ask me and I can teach you.

\subsection*{Note on collaboration on homework assignments}
\label{collaboration}

Group work on homework problems is allowed and encouraged.
Discussions are in general very
helpful and inspiring when learning mathematics.
Nevertheless, before talking to others, get well started
on the problems, and contribute your fair share to the process.

When completing the written homework assignments, everyone must write up his or her own
solutions in their own words.
It is very important that you truly understand the homework solutions you hand
in, otherwise you may be unpleasantly surprised by your in-class test results.

\subsection{Write-up tasks}

One of the goals of the course is to develop and improve
the skill of mathematical presentation and writing.
Therefore, the accuracy of mathematical writing
in homework and tests
is taken very seriously. You can get points off
if you do not explain your ideas clearly.
Typesetting your homework solutions in \TeX/\LaTeX{} is encouraged
but optional --- handwritten solutions are also fine.

Each week one of the students will be assigned the task of
writing down
detailed solutions to that week’s problem set (these write-ups are done
exclusively in \TeX/\LaTeX{}). 
These will be posted on the
course page after revision. The students' contributions will be evaluated and
will constitute a percentage of the final grade. Each student is expected
to contribute at least once.

\subsection{Midterm tests and the final exam}

The midterms and the final exam will feature
problems modeled after homework.
These tests are also very helpful as preparations 
for the analysis general exam.

\subsection{How to succeed in the course}

The best way to learn in the course is to come to all lectures, take good notes
(some notes will be provided),
ask many questions,
do all the homework problems, and express your solutions
clearly.
This will prepare you well for midterms and the final exam.

Mathematical questions are appreciated and encouraged any time during the
class. Please use the office hours as much as possible for additional
clarifications and occasional homework help.

\subsection{Grade distribution}

Your grade will consist of:
\begin{itemize}
	\item Homework --- 30\%, lowest homework dropped
	\item Midterms --- 15\% each
	\item Final exam --- 30\%
	\item Class participation, office hours discussion, write-ups --- 10\%
\end{itemize}
The score above 90\% is usually enough for an A.
The score below 50\% usually means failing.
Other factors such as in-class participation
and improvement over time may impact positively your final grade.













\section{Policies}

\subsection{Laptops and smartphones}

Please do not use laptops and smartphones during the class.
You won't need them to participate in the discussions, but they may easily distract
you or other students (or me!). If you \emph{absolutely} must use a laptop
(for typing up the lecture notes), please sit in the back row.

\subsection{Late/make up work} Each assignment will have due date and time.
Late assignments are not accepted. There will also be no make ups for the midterm tests and the final exam.
However, if you have special needs, emergency, or unavoidable conflicts, please
let me know as soon as possible, so we can arrange a workaround.

\subsection{Honor Code} The University of Virginia Honor Code applies to this
class and is taken seriously. Collaboration on homework
assignments is allowed within the bounds discussed above
in the corresponding section.
Any honor code violations will be referred to the
Honor Committee.

\subsection{Special needs}

All students with special needs requiring accommodations should present the
appropriate paperwork from the Student Disability Access Center (SDAC). It is
the student's responsibility to present this paperwork in a timely fashion and
follow up with the instructor about the accommodations being offered.
Accommodations for test-taking (e.g., extended time) should be arranged at
least 5 business days before an exam.




\end{document}
