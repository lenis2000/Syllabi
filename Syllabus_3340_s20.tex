\documentclass[oneside,11pt]{amsart}
\usepackage[utf8]{inputenc}%
\usepackage[english]{babel}%
\usepackage{amsmath,amssymb,amsthm,amsfonts}%
\usepackage[unicode]{hyperref}%
\usepackage{mathrsfs,bbm}%
\usepackage{paralist}
\usepackage{color}
\usepackage{longtable}
\usepackage{array}
\newcolumntype{L}[1]{>{\small\raggedright\arraybackslash}m{#1}}
\newcolumntype{T}[1]{>{\footnotesize\raggedright\arraybackslash}m{#1}}
\usepackage{stmaryrd}%
%\usepackage{refcheck}
\usepackage{graphicx}
\usepackage[DIV14]{typearea}
\usepackage{multicol,tikz}
\usepackage{datetime}
\usepackage{cleveref}

\usepackage[shadow]{todonotes}

\usepackage{etoolbox}
\patchcmd{\section}{\scshape}{\Large\itshape\bfseries}{}{}

\usepackage{caption}
\captionsetup{labelformat=empty,labelsep=none}

\hypersetup{
  colorlinks=true,
  linkcolor=blue!50!red,
  urlcolor=green!60!black
}

%%%%%%%%%%%%%%%%%%%%%%%%%%%%%%%%%%%%%%%%%%%%%%%%%%%%%%%%%%%%%%%%%%%%%%%%%%%%%%%%%%%%%%%%
\synctex=1
%%%%%%%%%%%%%%%%%%%%%%%%%%%%%%%%%%%%%%%%%%%%%%%%%%%%%%%%%%%%%%%%%%%%%%%%%%%%%%%%%%%%%%%%
%%%%%%%%%%%%%%%%%%%%%%%%%%%%%%%%%%%%%%%%%%%%%%%%%%%%%%%%%%%%%%%%%%%%%%%%%%%%%%%%%%%%%%%%
\newcommand{\score}[1]{\textit{#1}\addtocounter{totalscore}{#1}}
\newcommand{\razdel}[1]{\smallskip\underline{\textbf{#1:}}\smallskip}

\newcommand{\note}[1]{{\sf{}\color{blue}(#1)}}

\begin{document}

\title[MATH 3340: COMPLEX VARIABLES WITH APPLICATIONS]{MATH 3340: COMPLEX VARIABLES WITH APPLICATIONS}
\author{Leonid Petrov\\Spring 2020}
\date{Compiled on \today, \currenttime{} (in whatever timezone I was at that time).\\An up to date syllabus is always on \texttt{GitHub} at \url{https://github.com/lenis2000/Syllabi/blob/master/Syllabus_3340_s20.pdf}. For direct PDF download use \href{https://github.com/lenis2000/Syllabi/raw/master/Syllabus_3340_s20.pdf}{\texttt{this link}}.
	\LaTeX{} source with \textit{changes} to the syllabus is \href{https://github.com/lenis2000/Syllabi/blob/master/Syllabus_3340_s20.tex}{\texttt{here}}
(click ``History'').
\\Note that this PDF has green clickable links.}
\maketitle



\end{document}
