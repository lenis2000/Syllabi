\documentclass[oneside,11pt]{amsart}
\usepackage[utf8]{inputenc}%
\usepackage[english]{babel}%
\usepackage{amsmath,amssymb,amsthm,amsfonts}%
\usepackage[unicode]{hyperref}%
\usepackage{mathrsfs,bbm}%
\usepackage{paralist}
\usepackage{color}
\usepackage{longtable}
\usepackage{array}
\usepackage{stmaryrd}%
%\usepackage{refcheck}
\usepackage{graphicx}
\usepackage[DIV15]{typearea}
\usepackage{multicol,tikz}
\usepackage{datetime}
\usepackage{cleveref}

\usepackage[shadow]{todonotes}

\usepackage{etoolbox}
\patchcmd{\section}{\scshape}{\large\itshape\bfseries}{}{}

\usepackage{caption}
\captionsetup{labelformat=empty,labelsep=none}

\hypersetup{
  colorlinks=true,
  linkcolor=blue!50!red,
  urlcolor=green!60!black
}

%%%%%%%%%%%%%%%%%%%%%%%%%%%%%%%%%%%%%%%%%%%%%%%%%%%%%%%%%%%%%%%%%%%%%%%%%%%%%%%%%%%%%%%%
\synctex=1
%%%%%%%%%%%%%%%%%%%%%%%%%%%%%%%%%%%%%%%%%%%%%%%%%%%%%%%%%%%%%%%%%%%%%%%%%%%%%%%%%%%%%%%%
%%%%%%%%%%%%%%%%%%%%%%%%%%%%%%%%%%%%%%%%%%%%%%%%%%%%%%%%%%%%%%%%%%%%%%%%%%%%%%%%%%%%%%%%

\begin{document}

\title[Building Truth from Scratch]{EGMT 1520: Building Truth from Scratch\\(Empirical \& Scientific Engagement)}
\author{Leonid Petrov\\Fall 2025}
\date{Compiled on \today, \currenttime. An up to date syllabus is always at \href{https://github.com/lenis2000/Syllabi/blob/master/Syllabus_EGMT_f25.pdf}{\texttt{this link}}.}
\maketitle


\section{How do we know a claim is true?}

This course is a hands-on workshop in making and testing arguments in the context of mathematics.
We will generate conjectures from examples, search for counterexamples, and turn ideas into precise statements and proofs. Through problem-solving sessions and math debates, you'll practice evaluating arguments, giving and receiving constructive feedback, and communicating clearly in writing and speech. By experiencing mathematics as a creative process --- where patterns suggest conjectures and logical reasoning turns intuition into conviction --- you'll develop a practical sense for what counts as evidence in mathematics and how to build reliable conclusions.
By the end of the course, you will be able to
\begin{enumerate}[$\bullet$]
    \item \textbf{Define and delimit what constitutes valid mathematical evidence} by distinguishing between examples, counterexamples, conjectures, and formal proofs, while recognizing the limitations of empirical observations.
    % \emph{(Aligned with pillar objective: “define and delimit what constitutes empirical evidence”)}
    \item \textbf{Develop a framework for discerning different types of mathematical knowledge} by exploring how empirical evidence, abstract reasoning, and logical structure work together to shape mathematical understanding.
    % \emph{(Aligned with pillar objective: “develop a framework for discerning types of knowledge based on what is empirically
% observable in the natural, physical and social worlds”)}
    \item \textbf{Formulate and communicate mathematical reasoning} by translating intuitive insights into precise statements, evaluating the soundness of arguments, and engaging in constructive dialogue to identify and resolve reasoning gaps.
    % \emph{(Aligned with pillar objective: “evaluate supported claims about the natural and social worlds by framing empirical questions and methods and interpreting the claims in the context of new data”)}
    \item \textbf{Reflect on the nature of mathematical truth} by examining personal assumptions about certainty, analyzing when and why certain arguments are conclusive, and articulating how purely empirical approaches can both inform and limit our understanding of complex phenomena.
    % \emph{(Aligned with pillar objective: “articulate the limitations of using only empirical approaches to describe complex phenomena”)}
\end{enumerate}

\begin{center}
\vspace{1cm}
\includegraphics[width=.5\textwidth]{EGMT_image.png}
\vspace{1cm}
\end{center}

\newpage
\section{Contact \& Logistics}

\noindent
\begin{tabular}{ll}
\textbf{Instructor:} Leonid Petrov &\qquad \qquad \qquad\textbf{Office:} 209 Kerchof Hall \\
\textbf{Email:} \href{mailto:petrov@virginia.edu}{petrov@virginia.edu} & \qquad \qquad \qquad\textbf{Office hours:} TBA \\
\textbf{Website:} \url{https://lpetrov.cc}
\end{tabular}

\smallskip
\noindent\textbf{Weekly rhythm:} Class meets Mondays
3:30--6:00PM, Gilmer Hall 247
(September 1, 8, 15, 22, 29; October 6).
Your single weekly submission (two commonplace parts + one designated problem write-up) is due on \textbf{Sunday 10:00\,pm} on Canvas.

\smallskip
\noindent
\textbf{Math debate split session:} 
In addition, each student will attend one of two math debate sessions. I plan two
sessions, in the weeks of September 15 and 22. The timing will be 
determined by polling the class on their availability.

\section{How we work together}

This is a hands-on, pen-and-paper course. Each meeting begins with 
a brief review, then you receive
a fresh problem set. You work at your table in fixed small subgroups, develop ideas, test them, and revise when needed. I circulate, ask questions, and may invite nearby peers to listen in and challenge your reasoning.

\begin{enumerate}[$\bullet$]
  \item \textbf{Home subgroups (fixed):} On day one we form 6 ``home subgroups'' for in-class collaboration:
  $\delta, \ \theta, \ \zeta, \ \rho, \ \lambda, \ \phi$.\footnote{Math and science use Greek symbols very often. The names of the subgroups are 
	\emph{delta}, \emph{theta}, \emph{zeta}, \emph{rho}, \emph{lambda}, and \emph{phi}.}
	Subgroups remain stable.

  \item \textbf{In-class problem work (at the table):} At
	the start of class you receive a new problem set.
	Your group works at the table; when ready, you explain
	your solution to me. I act as a skeptical audience.
	You may use scrap paper or your commonplace book to write up your solution.
	During your subgroup's explanation, I may ask a different member to continue the explanation,
	so everyone should understand the full argument.

  \item \textbf{Random mixers:} 
		Near the end of class, we spend about 10-15 minutes 
		in randomly mixed groups.
		Quick reshuffles outside your home group allow you
		to compare partial solutions and share techniques.
		
	 \item \textbf{Mini-debate session (dedicated slot with a new
	 problem):} In a scheduled debate block (separate from
	 regular weekly work), the class receives a {new}
	 problem designed for argument-testing. There is
	 structured {work time} to develop approaches,
	 followed by {presentation and questioning}. 
	 A pair of presenters will be at the board explaining the
	 solution, while the rest of the class 
	 focuses on questioning and debating the claim.
  
	\item \textbf{Math debate:} Scheduled separately; see below for details.

	\item \textbf{Commonplace book:} Bring it every time.
	Use it for quick recaps, sketches, diagrams, partial
	proofs, and brief reflections. Photos of selected pages
	are submitted to Canvas each week (see details below).

  \item \textbf{Team roles (rotate each meeting):} In subgroups of $5$--$6$, 
		the following sample roles work well:
		\begin{compactitem}
			\item \textsc{Explainer}: articulates the current
			approach and restates the problem in your own words.
			\item \textsc{Skeptic}: presses on assumptions, looks
			for gaps, and frames precise questions.
			\item \textsc{Counterexample Hunter}: designs and
			tests examples/edge cases to probe the claim.
			\item \textsc{Recorder}: maintains a clean write-up in
			the commonplace book.
			\item \textsc{Verifier}: checks computations and logic, and
			ensures steps follow from the stated problem.
			\item \textsc{Connector}: ensures that all subgroup
			members understand and can explain the solution.
		\end{compactitem}
		You may keep the roles informal or explicitly assign them in your
		subgroup, but \emph{make sure you rotate them at least each week,
		if not more often}.

  \item \textbf{Devices:} Please keep devices in your bag 
		unless you have an accommodation that requires otherwise.
		This is a pen-and-paper course.
\end{enumerate}

\section{Commonplace notebook weekly assignments}

\begin{enumerate}[(1)]
  \item \textbf{What counted as evidence for me? (300 words / 1 page limit)}
	Name one math experience from this week
	and list \emph{exactly what} made it convincing \emph{to
	you} (e.g., a minimal example, a failed counterexample, a
	definition that removed ambiguity, an auxiliary statement
	that closed a gap). 
	End with one thing that would change your mind about 
	this experience being convincing.

  \item \textbf{Assumption audit (300 words / 1 page limit)}
	Pick one problem from the problem set, and list all
	assumptions you used in your solution
	(not just those stated in the problem). For each, mark:
	\emph{needed} / \emph{not needed} / \emph{uncertain}. Then
	try to \emph{drop one}: either give a tiny counterexample
	that shows it \emph{was} needed, or a brief note
	explaining why the argument still goes through.

  \item \textbf{Complete solution write-up (one problem)}
  Choose one problem \emph{designated for write-up} in the handout, and write a clear, self-contained solution. Define terms you use, justify each step, and cite any previously settled claims.
\end{enumerate}
\emph{Submission format:} Clear photos or a single PDF to Canvas by Sunday 10:00\,pm, submitted to the required assignment.

\section{Math debate}

The math debate is a structured two-team activity.
The teams consist of 8-9 students and are separate from the 
in-class subgroups. The class is split into four teams of 8-9 people,
two for each debate session.
The other teams are very welcome to attend the debate, too.
Splitting into teams will happen on September 15, the third class meeting.

The debate has the following structure:
\begin{enumerate}[$\bullet$]
\item \textbf{0:00--0:10} --- Welcome briefing and choosing the 
team that calls first by
a
1-minute mini-problem (any teammate may answer; the team, not a person, wins initiative). After Round~1, initiative alternates.

\item \textbf{0:10--1:10} | An hour of problem solving. Same packet 
is issued to both teams, and they collaborate in their own rooms or spaces.
Teams decide internally who will present which problem. 
Each team member can present at most once.
The teams also select a liaison for each round, who will announce the challenge.

\item \textbf{1:05--1:10} | Lock answers. Submit brief answer slips (claims/ideas only).

\item \textbf{1:10--2:20} |
Three rounds of debate.
Suppose Team~A has the initiative.
Current round liaison from Team~A
announces the problem.
There are two outcomes:
\begin{enumerate}
\item If Team~B accepts the challenge, they present their solution
(6 minutes max) $\to$
Team~A asks up to 2 questions (2 minutes max)
$\to$
Team~B then presents their opposition (5 minutes max)
$\to$
Team~A replies to the opposition (3 minutes max).
\item If Team~B rejects the challenge, roles change, and Team~A presents their solution (6 minutes max) $\to$
Team~B asks up to 2 questions (2 minutes max)
$\to$
Team~A then presents their opposition (5 minutes max)
$\to$
Team~B replies to the opposition (3 minutes max).
\end{enumerate}
For each round, there is one person from each team at the board. Each person
may be at the board at most once per the whole debate.
In each round, every team has a 30-sec timeout to consult with their presenter. Outside of the timeout, the teams and the audience must be silent.

\item Scoring is 12 points per round,
split between the two teams and ``the audience'':\footnote{Team scores 
in the math debate
don't affect the course grade.}
\begin{enumerate}
\item \textbf{Presenting team (0--12):} Correctness \& completeness (0--6); 
  Structure \& clarity at board (0--4) and presents a correct solution, the opposing team may earn full 12 points(0--2).
  \item \textbf{Opposing team (0--6):} Validity \& precision of objections (0--4);
  Quality of questions (0--2).
  \item \textbf{No-solution claim:} If the opponent \emph{demonstrates} a decisive gap that cannot be patched within time and presents a correct solution, the opposing team may earn full 12 points.
\end{enumerate}

\item \textbf{2:20--2:30} | Wrap-up.
\end{enumerate}



\section{Engaging Grounds}

TBD after August 19.

\section{Grading}

\begin{enumerate}[$\bullet$]
\item \textbf{In-class work \& explanations (35\%):}
Assessed on:
\begin{compactitem}
  \item \textbf{Clarity} of explanations at the table.
  \item \textbf{Responsiveness} to questions and proposed counterexamples (can you repair, refine, or retract a claim when pressed?).
  \item \textbf{Equitable participation} within the subgroup (roles rotate; multiple voices contribute to each explanation).
\end{compactitem}
Attendance is required to earn credit. A problem is ``accepted'' when your subgroup can answer follow-ups and defend key steps without unresolved gaps.

 \item \textbf{Commonplace notebook (35\%):}
	A three-part weekly writing assignment, submitted
	on Canvas by Sunday 10:00\,pm as photos or a PDF scan. 
	Grading is for each of the three parts on the following scale:
	\emph{Meets} (full credit), 
	\emph{Revise} 
	(resubmission within 72 hours yields full credit), or
	\emph{Missing} (0).
	Lowest one weekly submission is dropped
	automatically.

  \item \textbf{Math debate (20\%):}
	Attendance and participation in the math debate.

	\item \textbf{Engaging Grounds (10\%):}
	TBD after August 19.
\end{enumerate}

The percentage grade is \emph{not rounded up}, and is calculated 
according to the following scale:
\begin{center}
\begin{tabular}{ll|ll|ll}
100+ & A+ & 83--86.99 & B & 67--69.99 & D+ \\
93--99.9 & A & 80--82.99 & B- & 63--66.99 & D \\
90--92.99 & A- & 77--79.99 & C+ & 60--62.99 & D- \\
87--89.99 & B+ & 73--76.99 & C & $<60$ & F \\
70--72.99 & C- & & \\
\end{tabular}
\end{center}


\section{Policies}

\subsection{Late work}

Each weekly notebook assignment is due on Sunday at 10pm.
You have one no-questions-asked Grace Week: submit any one weekly notebook by Wednesday 10 pm with no penalty. Beyond that,
late assignments are not accepted. 
If you have special needs or an emergency, please let me know as soon as possible.

\subsection{Honor Code}
The University of Virginia Honor Code applies to this course and is taken seriously. Any Honor Code violations will be referred to the Honor Committee. 

\subsection{Devices}

This is a pen-and-paper course. Please keep all devices in your bag during class activities, unless you have an accommodation that requires otherwise.

\subsection{Use of AI}

\begin{enumerate}[$\bullet$]
  \item \textbf{Allowed for learning}: asking conceptual questions; generating small practice examples; checking ideas (not text) for plausibility.

\noindent
For learning with LLMs, here is an example \href{https://gist.githubusercontent.com/lenis2000/cb5ea004f8aa6461be71398e19ae488e/raw/a0103eab0b865a1cedf2f3bf3c00d217bd294005/AI_hint_prompt.txt}{\texttt{prompt}} that might be helpful.
  \item \textbf{Allowed with citation}: grammar/clarity edits to your own draft (light copy-editing only). If used, add a one-line note at the end: AI-assisted for copy-editing only.

  \item \textbf{Not allowed}: drafting any part of the submitted solution or reflections; step-by-step proofs; problem-specific hints beyond what's provided in class; rewriting your math into polished prose.
\end{enumerate}



\subsection{Attendance}

Attendance is expected at all class meetings, including the extra meeting for math debate. If you miss a class, you are responsible for catching up on the material covered. If you have a valid reason for missing a class, please inform me in advance.

\subsection{Special needs}

All students with special needs requiring accommodations should present the appropriate paperwork from the Student Disability Access Center (SDAC). It is the student's responsibility to present this paperwork in a timely fashion and to follow up with the instructor about the accommodations being offered.
% \subsection{Recording for a personal study}
%
% Class sessions for this course may be audio recorded as a
% reasonable accommodation for a disability for the student's
% own personal study and review. These audio recordings will
% be deleted at the end of the semester. Recordings will not
% be reproduced, shared with those not enrolled in the class,
% nor uploaded to other online environments.



\end{document}
