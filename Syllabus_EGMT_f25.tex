\documentclass[oneside,11pt]{amsart}
\usepackage[utf8]{inputenc}%
\usepackage[english]{babel}%
\usepackage{amsmath,amssymb,amsthm,amsfonts}%
\usepackage[unicode]{hyperref}%
\usepackage{mathrsfs,bbm}%
\usepackage{paralist}
\usepackage{color}
\usepackage{longtable}
\usepackage{array}
\newcolumntype{L}[1]{>{\small\raggedright\arraybackslash}m{#1}}
\newcolumntype{T}[1]{>{\footnotesize\raggedright\arraybackslash}m{#1}}
\usepackage{stmaryrd}%
%\usepackage{refcheck}
\usepackage{graphicx}
\usepackage[DIV15]{typearea}
\usepackage{multicol,tikz}
\usepackage{datetime}
\usepackage{cleveref}

\usepackage[shadow]{todonotes}

\usepackage{etoolbox}
\patchcmd{\section}{\scshape}{\Large\itshape\bfseries}{}{}

\usepackage{caption}
\captionsetup{labelformat=empty,labelsep=none}

\hypersetup{
  colorlinks=true,
  linkcolor=blue!50!red,
  urlcolor=green!60!black
}

%%%%%%%%%%%%%%%%%%%%%%%%%%%%%%%%%%%%%%%%%%%%%%%%%%%%%%%%%%%%%%%%%%%%%%%%%%%%%%%%%%%%%%%%
\synctex=1
%%%%%%%%%%%%%%%%%%%%%%%%%%%%%%%%%%%%%%%%%%%%%%%%%%%%%%%%%%%%%%%%%%%%%%%%%%%%%%%%%%%%%%%%
%%%%%%%%%%%%%%%%%%%%%%%%%%%%%%%%%%%%%%%%%%%%%%%%%%%%%%%%%%%%%%%%%%%%%%%%%%%%%%%%%%%%%%%%
\newcommand{\score}[1]{\textit{#1}\addtocounter{totalscore}{#1}}
\newcommand{\razdel}[1]{\smallskip\underline{\textbf{#1:}}\smallskip}

\newcommand{\note}[1]{{\sf{}\color{blue}(#1)}}

\begin{document}

\title[Building Truth from Scratch]{EGMT 1520: Building Truth from Scratch\\(Empirical \& Scientific Engagement)}
\author{Leonid Petrov\\Fall 2025}
\date{Compiled on \today, \currenttime. An up to date syllabus is always at \href{https://github.com/lenis2000/Syllabi/blob/master/Syllabus_EGMT_f25.pdf}{\texttt{this link}}.}
\maketitle

\bigskip


\section*{How do we know a claim is true?}

This course is a hands-on workshop in making and testing arguments in the context of mathematics.
We will generate conjectures from examples, search for counterexamples, and turn ideas into precise statements and proofs. Through problem-solving sessions and math debates, you'll practice evaluating arguments, giving and receiving constructive feedback, and communicating clearly in writing and speech. By experiencing mathematics as a creative process --- where patterns suggest conjectures and logical reasoning turns intuition into conviction --- you'll develop a practical sense for what counts as evidence in mathematics and how to build reliable conclusions.
By the end of the course, you will be able to
\begin{enumerate}
    \item \textbf{Define and delimit what constitutes valid mathematical evidence} by distinguishing between examples, counterexamples, conjectures, and formal proofs, while recognizing the limitations of empirical observations.
    % \emph{(Aligned with pillar objective: “define and delimit what constitutes empirical evidence”)}
    \item \textbf{Develop a framework for discerning different types of mathematical knowledge} by exploring how empirical evidence, abstract reasoning, and logical structure work together to shape mathematical understanding.
    % \emph{(Aligned with pillar objective: “develop a framework for discerning types of knowledge based on what is empirically
% observable in the natural, physical and social worlds”)}
    \item \textbf{Formulate and communicate mathematical reasoning} by translating intuitive insights into precise statements, evaluating the soundness of arguments, and engaging in constructive dialogue to identify and resolve reasoning gaps.
    % \emph{(Aligned with pillar objective: “evaluate supported claims about the natural and social worlds by framing empirical questions and methods and interpreting the claims in the context of new data”)}
    \item \textbf{Reflect on the nature of mathematical truth} by examining personal assumptions about certainty, analyzing when and why certain arguments are conclusive, and articulating how purely empirical approaches can both inform and limit our understanding of complex phenomena.
    % \emph{(Aligned with pillar objective: “articulate the limitations of using only empirical approaches to describe complex phenomena”)}
\end{enumerate}




\end{document}
